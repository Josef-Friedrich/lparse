\documentclass{ltxdoc}

\EnableCrossrefs
\CodelineIndex
\RecordChanges

\usepackage{mdframed}
\usepackage{minted}
\usepackage{multicol}
\usepackage{luacode}
\usepackage{syntax}
\usepackage{color}

\definecolor{gray}{gray}{0.75}

\usemintedstyle{friendly}
\BeforeBeginEnvironment{minted}{\begin{mdframed}}
\AfterEndEnvironment{minted}{\end{mdframed}}
\setminted{
  breaklines=true,
  fontsize=\footnotesize,
  style=manni,
}
\def\lua#1{\mintinline{lua}|#1|}
\def\latex#1{\mintinline{latex}|#1|}

% File names must not contain underscores
\NewDocumentCommand { \InputLatex } { O{} m } {
  \begin{mdframed}
  \inputminted[linenos=false,#1]{latex}{examples/#2}
  \end{mdframed}
}

\NewDocumentCommand { \InputTex } { O{} m } {
  \begin{mdframed}
  \inputminted[linenos=false,#1]{tex}{examples/#2}
  \end{mdframed}
}

\NewDocumentCommand { \InputLua } { O{} m } {
  \begin{mdframed}
  \inputminted[linenos=false,#1]{lua}{examples/#2}
  \end{mdframed}
}

\begin{document}

\providecommand*{\url}{\texttt}

\title{The \textsf{lparse} package}
\author{%
  Josef Friedrich\\%
  \url{josef@friedrich.rocks}\\%
  \href{https://github.com/Josef-Friedrich/lparse}
       {github.com/Josef-Friedrich/lparse}%
}
\date{0.2.0 from 2025/06/19}

\maketitle

\InputLatex[firstline=5,lastline=14]{lualatex/title-page.tex}

\newpage

\tableofcontents

\newpage

% \section{Einführung}
\section{Introduction}

% Der Name |lparse| ist von |xparse| abgeleitet.
The name |lparse| is derived from |xparse|.
%
% Das |x| wurde durch |l| ersetzt, da dieses Packet nur zusammen mit
% Lua\TeX{} funktioniert.
The |x| has been replaced by |l| because this package only works with Lua\TeX{}.
%
% |l| steht für \emph{Lua}
|l| stands for \emph{Lua}.
%
% Ebenso wie bei dem Paket |xparse| ist es möglich, die Argumente eines
% Makros mit Hilfe einer speziellen Syntax bestehend aus einzelnen
% Buchstaben zu beschreiben.
Just as with |xparse|, it is possible to use a special syntax consisting
of single letters to express the arguments of a macro.
%
% |lparse| ist jedoch im Stande unabhängig von der verwendeten
% Makrosammlung - ob LaTeX oder ConTeXt oder sogar Plain TeX - Argumente
% einzulesen.
However, |lparse| is able to read arguments regardless of the macro
systemd used - whether \LaTeX{} or Con\TeX{}t or even plain \TeX.
%
% Als Engine musst natürlich immer LuaTeX verwendet werden.
Of course, Lua\TeX{} must always be used as the engine.

\subsection*{Similar projects}

% Für ConTeXt gibt es einen ähnlichen Argumenten scanner.
For Con\TeX{}t there is a similar argument scanner (see Con\TeX{}t Lua
Document
\href{http://www.pragma-ade.nl/general/manuals/cld-mkiv.pdf}{cld-mkiv}).
%
% Dieser Scanner ist in folgenden Dateien implementiert
This scanner is implemented in the following files:
\href{https://github.com/contextgarden/context/blob/main/tex/context/base/mkiv/toks-scn.lua}{toks-scn.lua}
\href{https://github.com/contextgarden/context/blob/main/tex/context/base/mkiv/toks-aux.lua}{toks-aux.lua}
\href{https://github.com/contextgarden/context/blob/main/tex/context/base/mkiv/toks-ini.lua}{toks-ini.lua}
% Con\TeX{}t scanner verwendet offenbar die Token Bibliothek des LuaTeX Nachfolgeprojekts luametaTeX
%
Con\TeX{}t scanner apparently uses the token library of the LuaTeX
successor project luameta\TeX:
\href{https://github.com/contextgarden/context/blob/main/source/luametatex/source/lua/lmttokenlib.c}{lmttokenlib.c}

\section{The Lua API of \texttt{lparse}}

\subsection{Description of the argument specification}
%
% Folgende Listen, die die Argumenttypen beschreiben, entstammen den
% Handbücher usrguide bzw. xparse.
The following lists describing the argument types are
taken from the manuals
\href{http://mirrors.ctan.org/macros/latex/base/usrguide.pdf}{usrguide}
and
\href{http://mirrors.ctan.org/macros/latex/contrib/l3packages/xparse.pdf}{xparse}.
%
% Die Beschreibungstexte der einzelnen Argumententypen wurden nur leicht angepasst.
The descriptive texts of the individual argument types have only been
slightly adjusted.
%
% Die noch nicht unterstützen Argumentypen sind eingeklammert.
The argument types that are not yet supported are bracketed.

% /usr/local/texlive/texmf-dist/source/latex/l3packages/xparse/xparse.dtx
\begin{itemize}
  \item[m] A standard mandatory argument, which can either be a single
    token alone or multiple tokens surrounded by curly braces |{}|.
    Regardless of the input, the argument will be passed to the
    internal code without the outer braces. This is the \pkg{lparse}
    type specifier for a normal \TeX{} argument.

  \item[r] Given as \texttt{r}\meta{token1}\meta{token2}, this denotes a
    \enquote{required} delimited argument, where the delimiters are
    \meta{token1} and \meta{token2}. If the opening delimiter
    \meta{token1} is missing, |nil| will be
    returned after a suitable error.

  \item[R] Given as \texttt{R}\meta{token1}\meta{token2}\marg{default},
    this is a \enquote{required} delimited argument as for~\texttt{r},
    but it has a user-definable recovery \meta{default} instead of
    |nil|.

  \item[v] Reads an argument \enquote{verbatim}, between the following
    character and its next occurrence.

  \color{gray}

  \item[(b)] Not implemented! Only suitable in the argument specification of an
    environment, it denotes the body of the environment, between
    |\begin|\marg{environment} and |\end|\marg{environment}.

\end{itemize}
The types which define optional arguments are:
\begin{itemize}
  \item[o] A standard \LaTeX{} optional argument, surrounded with square
    brackets, which will supply
    |nil| if not given (as described later).

  \item[d] Given as \texttt{d}\meta{token1}\meta{token2}, an optional
    argument which is delimited by \meta{token1} and \meta{token2}.
    As with \texttt{o}, if no
    value is given |nil| is returned.

  \item[O] Given as \texttt{O}\marg{default}, is like \texttt{o}, but
    returns \meta{default} if no value is given.

  \item[D] Given as \texttt{D}\meta{token1}\meta{token2}\marg{default},
    it is as for \texttt{d}, but returns \meta{default} if no value is given.
    Internally, the \texttt{o}, \texttt{d} and \texttt{O} types are
    short-cuts to an appropriated-constructed \texttt{D} type argument.

  \item[s] An optional star, which will result in a value
    |true| if a star is present and |false|
    otherwise (as described later).

  \item[t] An optional \meta{token}, which will result in a value
    |true| if \meta{token} is present and |false|
    otherwise. Given as \texttt{t}\meta{token}.

  \color{gray}

  \item[(e)] Not implemented! Given as \texttt{e}\marg{tokens}, a set of optional
    \emph{embellishments}, each of which requires a \emph{value}.
    If an embellishment is not present, |-NoValue-| is returned.  Each
    embellishment gives one argument, ordered as for the list of
    \meta{tokens} in the argument specification.  All \meta{tokens}
    must be distinct.  \emph{This is an experimental type}.
  \item[(E)] Not implemented! As for \texttt{e} but returns one or more \meta{defaults}
    if values are not given: \texttt{E}\marg{tokens}\marg{defaults}.
\end{itemize}

\subsection{Function: \texttt{scan}}

\InputTex{luatex/function-scan.tex}

\subsection{Class: \texttt{Scanner}}

\subsubsection{Method: \texttt{Scanner:scan()}}

\subsubsection{Method: \texttt{Scanner:export()}}

\subsubsection{Method: \texttt{Scanner:assert()}}

\subsubsection{Method: \texttt{Scanner:debug()}}

% Hilfsfunktionen
\subsection{Auxiliary functions}

%
% Einige Hilfsfunktionen werden in der Tabelle utils exportiert.
Some auxiliary functions are exported in the \lua{utils} table:

\begin{minted}{lua}
local lparse = require('lparse')

local parse_spec = lparse.utils.parse_spec
local scan_oarg = lparse.utils.scan_oarg
\end{minted}

%%
%
%%

\subsubsection*{Function: \texttt{utils.scan_oarg(init_delim, end_delim)}}

% Plain \TeX{} kennt keine optionalen Argumente (oarg).
Plain \TeX{} does not know optional arguments \oarg{oarg}.
%
% Die Funktion ermöglicht es nicht nur in \LaTeX{}, sondern auch in
% Plain \TeX{} nach optionalen Argumenten zu suchen.
The function \lua{scan_oarg} allows to search for optional arguments not only
in \LaTeX{} but also in Plain \TeX.
%
% Die Funktion basiert auf der Token-Bibliothek.
The function uses the token library built into Lua\TeX{}.
%
% Die beiden Parameter \lua{initial_delimiter} und \lua{end_delimiter}
% können weggelassen werden.
The two parameters \lua{init_delim} and \lua{end_delim} can
be omitted.
%
% Dann werden eckige Klammern als Begrenzungszeichen angenommen.
Then square brackets are assumed to be delimiters.
%
% \lua{utils.scan_oarg('(', ')')} sucht beispielsweise nach an einem
% optionalen Argument in runden Klammern.
\lua{utils.scan_oarg(‘(’, ‘)’)} searches for an optional argument in
round brackets, for example.
%
% Die Funktion gibt die Zeichenkette zwischen den Begrenzungszeichen
% zurück, oder nil wenn Begrenzungszeichen gefunden werden konnten.
The function returns the string between the delimiters or \lua{nil} if
no delimiters could be found.
%
% Die Begrenzungszeichen sind im Ergebnis nicht enthalten.
The delimiters themselves are not included in the result.
%
% Nach dem \latex{\directlua{}} darf das Makro, indem
% \lua{utils.scan_oarg} eingesetzt wird, zu keinen Zeichen expandieren.
After the \latex{\directlua{}}, the macro using \lua{scan_oarg}
must not expand to any characters.

\InputTex{luatex/function-utils.scan-oarg.tex}

\section{Known limitations}

% Im einigen Lua\LaTeX{} Umgebungen funktioniert das Scannen nach
% optionalen Argument (oargs) nicht.
In some Lua\LaTeX{} environments, scanning for optional arguments
(oargs) does not work.
% Ich weiß leider nicht warum.
Unfortunately I don't know why.
% Über Hilfe wäre ich sehr dankbar.
I would be very grateful for help.

\subsection{In tabular environment}

% Steht in einer |tabular|-Umgebung vor einem Kommando, der mit |lparse|
% gescannt wird, ein beliebert Text funktioniert es nicht, sonst schon.
In a |tabular| environment, if a command that is scanned with |lparse| is
preceded by any text, it does not work, otherwise it does.

\InputTex{lualatex/error-tabular.tex}

\subsection{In multline environment}

\InputTex{lualatex/error-multline.tex}

\clearpage

\section{Implementation}

%%
%
%%

\subsection{lparse.lua}

\inputminted[linenos=true]{lua}{lparse.lua}

%%
%
%%

\clearpage

\subsection{lparse.tex}

\inputminted[linenos=true]{latex}{lparse.tex}

\clearpage

\subsection{lparse.sty}

\inputminted[linenos=true]{latex}{lparse.sty}

\pagebreak
\PrintIndex
\end{document}
