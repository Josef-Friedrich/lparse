\documentclass{ltxdoc}

\EnableCrossrefs
\CodelineIndex
\RecordChanges

\usepackage{mdframed}
\usepackage{minted}
\usepackage{multicol}
\usepackage{luacode}
\usepackage{syntax}

\usemintedstyle{friendly}
\BeforeBeginEnvironment{minted}{\begin{mdframed}}
\AfterEndEnvironment{minted}{\end{mdframed}}
\setminted{
  breaklines=true,
  fontsize=\footnotesize,
  style=manni,
}
\def\lua#1{\mintinline{lua}|#1|}
\def\latex#1{\mintinline{latex}|#1|}

\NewDocumentCommand { \InputLatex } { O{} m } {
  \begin{mdframed}
  \inputminted[linenos=false,#1]{latex}{examples/#2}
  \end{mdframed}
}

\NewDocumentCommand { \InputLua } { O{} m } {
  \begin{mdframed}
  \inputminted[linenos=false,#1]{lua}{examples/#2}
  \end{mdframed}
}

\begin{document}

\providecommand*{\url}{\texttt}

\title{The \textsf{lparse} package}
\author{%
  Josef Friedrich\\%
  \url{josef@friedrich.rocks}\\%
  \href{https://github.com/Josef-Friedrich/lparse}
       {github.com/Josef-Friedrich/lparse}%
}
\date{0.1.0 from 2022/12/30}

\maketitle

\InputLatex[firstline=5,lastline=14]{title-page.tex}

\newpage

\tableofcontents

\newpage

% \section{Einführung}
\section{Introduction}

\href{http://www.pragma-ade.nl/general/manuals/cld-mkiv.pdf}{cld-mkiv}

\href{https://github.com/contextgarden/context/blob/main/tex/context/base/mkiv/toks-scn.lua}{toks-scn.lua}
\href{https://github.com/contextgarden/context/blob/main/tex/context/base/mkiv/toks-aux.lua}{toks-aux.lua}
\href{https://github.com/contextgarden/context/blob/main/tex/context/base/mkiv/toks-ini.lua}{toks-ini.lua}

\href{https://github.com/contextgarden/context/blob/main/source/luametatex/source/lua/lmttokenlib.c}{lmttokenlib.c}

\href{http://mirrors.ctan.org/macros/latex/base/usrguide.pdf}{usrguide}

\href{http://mirrors.ctan.org/macros/latex/contrib/l3packages/xparse.pdf}{xparse}

% /usr/local/texlive/texmf-dist/source/latex/l3packages/xparse/xparse.dtx
\begin{itemize}
  \item[m] A standard mandatory argument, which can either be a single
    token alone or multiple tokens surrounded by curly braces |{}|.
    Regardless of the input, the argument will be passed to the
    internal code without the outer braces. This is the \pkg{xparse}
    type specifier for a normal \TeX{} argument.
  \item[r] Given as \texttt{r}\meta{token1}\meta{token2}, this denotes a
    \enquote{required} delimited argument, where the delimiters are
    \meta{token1} and \meta{token2}. If the opening delimiter
    \meta{token1} is missing, |nil| will be
    returned after a suitable error.
  \item[R] Given as \texttt{R}\meta{token1}\meta{token2}\marg{default},
    this is a \enquote{required} delimited argument as for~\texttt{r},
    but it has a user-definable recovery \meta{default} instead of
    |nil|.
  \item[v] Reads an argument \enquote{verbatim}, between the following
    character and its next occurrence, in a way similar to the argument
    of the \LaTeXe{} command \cs{verb}. Thus a \texttt{v}-type argument
    is read between two identical characters, which cannot be any of |%|, |\|,
    |#|, |{|, |}| or \verb*| |.
    The verbatim argument can also be enclosed between braces, |{| and |}|.
    A command with a verbatim
    argument will produce an error when it appears within an argument of
    another function.
  \item[(b)] Not implemented! Only suitable in the argument specification of an
    environment, it denotes the body of the environment, between
    |\begin|\marg{environment} and |\end|\marg{environment}.
\end{itemize}
The types which define optional arguments are:
\begin{itemize}
  \item[o] A standard \LaTeX{} optional argument, surrounded with square
    brackets, which will supply
    |nil| if not given (as described later).
  \item[d] Given as \texttt{d}\meta{token1}\meta{token2}, an optional
    argument which is delimited by \meta{token1} and \meta{token2}.
    As with \texttt{o}, if no
    value is given |nil| is returned.
  \item[O] Given as \texttt{O}\marg{default}, is like \texttt{o}, but
    returns \meta{default} if no value is given.
  \item[D] Given as \texttt{D}\meta{token1}\meta{token2}\marg{default},
    it is as for \texttt{d}, but returns \meta{default} if no value is given.
    Internally, the \texttt{o}, \texttt{d} and \texttt{O} types are
    short-cuts to an appropriated-constructed \texttt{D} type argument.
  \item[s] An optional star, which will result in a value
    |true| if a star is present and |false|
    otherwise (as described later).
  \item[t] An optional \meta{token}, which will result in a value
    |true| if \meta{token} is present and |false|
    otherwise. Given as \texttt{t}\meta{token}.
  \item[(e)] Not implemented! Given as \texttt{e}\marg{tokens}, a set of optional
    \emph{embellishments}, each of which requires a \emph{value}.
    If an embellishment is not present, |-NoValue-| is returned.  Each
    embellishment gives one argument, ordered as for the list of
    \meta{tokens} in the argument specification.  All \meta{tokens}
    must be distinct.  \emph{This is an experimental type}.
  \item[(E)] Not implemented! As for \texttt{e} but returns one or more \meta{defaults}
    if values are not given: \texttt{E}\marg{tokens}\marg{defaults}.
\end{itemize}

\clearpage

\section{Implementation}

%%
%
%%

\subsection{lparse.lua}

\inputminted[linenos=true]{lua}{lparse.lua}

%%
%
%%

\clearpage

\subsection{lparse.tex}

\inputminted[linenos=true]{latex}{lparse.tex}

\pagebreak
\PrintIndex
\end{document}
